% Options for packages loaded elsewhere
\PassOptionsToPackage{unicode}{hyperref}
\PassOptionsToPackage{hyphens}{url}
\documentclass[
]{article}
\usepackage{xcolor}
\usepackage{amsmath,amssymb}
\setcounter{secnumdepth}{-\maxdimen} % remove section numbering
\usepackage{iftex}
\ifPDFTeX
  \usepackage[T1]{fontenc}
  \usepackage[utf8]{inputenc}
  \usepackage{textcomp} % provide euro and other symbols
\else % if luatex or xetex
  \usepackage{unicode-math} % this also loads fontspec
  \defaultfontfeatures{Scale=MatchLowercase}
  \defaultfontfeatures[\rmfamily]{Ligatures=TeX,Scale=1}
\fi
\usepackage{lmodern}
\ifPDFTeX\else
  % xetex/luatex font selection
\fi
% Use upquote if available, for straight quotes in verbatim environments
\IfFileExists{upquote.sty}{\usepackage{upquote}}{}
\IfFileExists{microtype.sty}{% use microtype if available
  \usepackage[]{microtype}
  \UseMicrotypeSet[protrusion]{basicmath} % disable protrusion for tt fonts
}{}
\makeatletter
\@ifundefined{KOMAClassName}{% if non-KOMA class
  \IfFileExists{parskip.sty}{%
    \usepackage{parskip}
  }{% else
    \setlength{\parindent}{0pt}
    \setlength{\parskip}{6pt plus 2pt minus 1pt}}
}{% if KOMA class
  \KOMAoptions{parskip=half}}
\makeatother
\ifLuaTeX
\usepackage[bidi=basic]{babel}
\else
\usepackage[bidi=default]{babel}
\fi
\babelprovide[main,import]{french}
% get rid of language-specific shorthands (see #6817):
\let\LanguageShortHands\languageshorthands
\def\languageshorthands#1{}
\setlength{\emergencystretch}{3em} % prevent overfull lines
\providecommand{\tightlist}{%
  \setlength{\itemsep}{0pt}\setlength{\parskip}{0pt}}
\usepackage{bookmark}
\IfFileExists{xurl.sty}{\usepackage{xurl}}{} % add URL line breaks if available
\urlstyle{same}
\hypersetup{
  pdftitle={À propos des auteurs -- Traitement d\textquotesingle images satellites avec Python},
  pdflang={fr},
  hidelinks,
  pdfcreator={LaTeX via pandoc}}

\title{À propos des auteurs -- Traitement d\textquotesingle images
satellites avec Python}
\author{}
\date{}

\begin{document}
\maketitle

\phantomsection\label{quarto-document-content}
\phantomsection\label{title-block-header}
\section{\texorpdfstring{\protect\hypertarget{auteurs}{}{À propos des
auteurs}}{À propos des auteurs}}\label{uxe0-propos-des-auteurs}

\href{https://www.usherbrooke.ca/recherche/fr/specialistes/details/samuel.foucher}{\textbf{Samuel
Foucher}} est professeur au
\href{https://www.usherbrooke.ca/geomatique/}{Département de géomatique
appliquée} de l'\href{https://www.usherbrooke.ca/}{Université de
Sherbrooke}. Il y enseigne aux
\href{https://www.usherbrooke.ca/geomatique/etudes/programmes}{programmes
de 1\textsuperscript{er} et 2\textsuperscript{e} cycles de géomatique}
les cours \emph{Traitement numérique des images de télédétection},
\emph{Base de données géospatiales} et \emph{Apprentissage profond
appliqué à l'observation de la Terre}. Ses intérêts de recherche portent
sur le traitement d'images et l'application de l'IA aux données
géospatiales.

\href{https://www.usherbrooke.ca/recherche/fr/specialistes/details/philippe.apparicio}{\textbf{Philippe
Apparicio}} est professeur titulaire au
\href{https://www.usherbrooke.ca/geomatique/}{Département de géomatique
appliquée} de l'\href{https://www.usherbrooke.ca/}{Université de
Sherbrooke}. Il y enseigne aux
\href{https://www.usherbrooke.ca/geomatique/etudes/programmes}{programmes
de 1\textsuperscript{er} et 2\textsuperscript{e} cycles de géomatique}
les cours \emph{Transport et mobilité durable}, \emph{Modélisation et
analyse spatiale} et \emph{Géomatique appliquée à la gestion urbaine}.
Durant les dernières années, il a offert plusieurs formations aux Écoles
d'été du Centre interuniversitaire québécois de statistiques sociales
(\href{https://www.ciqss.org/}{CIQSS}). Géographe de formation, ses
intérêts de recherche incluent la justice et l'équité environnementale,
la mobilité durable, les pollutions atmosphérique et sonore, et le vélo
en ville. Il a publié une centaine d'articles scientifiques dans
différents domaines des études urbaines et de la géographie mobilisant
la géomatique et l'analyse spatiale.

\href{https://www.usherbrooke.ca/geomatique/departement/personnel/personnel-enseignant/mickael-germain}{\textbf{Mickaël
Germain}} est professeur agrégé au
\href{https://www.usherbrooke.ca/geomatique/}{Département de géomatique
appliquée} de l'\href{https://www.usherbrooke.ca/}{Université de
Sherbrooke}.

\href{https://www.usherbrooke.ca/geomatique/departement/personnel/personnel-enseignant/yacine-bouroubi}{\textbf{Yacine
Bouroubi}} est professeur agrégé au
\href{https://www.usherbrooke.ca/geomatique/}{Département de géomatique
appliquée} de l'\href{https://www.usherbrooke.ca/}{Université de
Sherbrooke}.

\href{https://www.linkedin.com/in/\%C3\%A9tienne-clabaut-793a4176/?originalSubdomain=ca}{\textbf{Étienne
Clabaut}} est professeur associé au
\href{https://www.usherbrooke.ca/geomatique/}{Département de géomatique
appliquée} de l'\href{https://www.usherbrooke.ca/}{Université de
Sherbrooke}.

\end{document}
